% Header mit Deklarationen
\documentclass[%
	pdftex,%              PDFTex verwenden
	a4paper,%             A4 Papier
	twoside,%             Einseitig/ wzeiseitig
	bibtotoc,%    		Literaturverzeichnis einfügen bibtotocnumbered: nummeriert
	liststotoc,%		Verzeichnisse einbinden in toc
	idxtotoc,%            Index ins Verzeichnis einfügen
	halfparskip,%        Europäischer Satz mit abstand zwischen Absätzen
	%chapterprefix,%       Kapitel anschreiben als Kapitel
	headsepline,%         Linie nach Kopfzeile
	footsepline,%         Linie vor Fusszeile
	pointlessnumbers,%     Nummern ohne abschließenden Punkt
	12pt,%                 Grössere Schrift, besser lesbar am bildschrim
	bibliography=totoc,
	cleardoublepage=empty,
	index=totoc,
	listof=totoc,
]{scrreprt}

%Maxi Biblio-Management
\usepackage[square,numbers]{natbib}
\bibliographystyle{unsrtdin}

%Zindath Biliothek Management
%\usepackage[backend=bibtex,bibencoding=ascii,style=numeric,citestyle=numeric-comp,defernumbers=true]{biblatex} 

%
% Seitenränder
%
\usepackage{geometry}
\geometry{a4paper, top=30mm, left=30mm, right=20mm, bottom=25mm, headsep=10mm, footskip=10mm}



%
% Paket für Übersetzungen ins Deutsche
%
\usepackage[french,english,ngerman]{babel}

%
% Pakete um Latin1 Zeichnensätze verwenden zu können und die dazu
% passenden Schriften.
%
%\usepackage[latin1]{inputenc}
% UTF8 Kompatabilität
\usepackage[utf8]{inputenc}
\usepackage[T1]{fontenc}

%
% Paket für Quotes
%
\usepackage[babel,french=guillemets,german=quotes]{csquotes}

%
% Paket zum Erweitern der Tabelleneigenschaften
%
\usepackage{array}


%
% Paket für schönere Tabellen
%
\usepackage{booktabs}

%
% Paket um Grafiken einbetten zu können
%
\usepackage{graphicx}
\usepackage{subfigure}
\usepackage{float} % zum abstellen der float umgebung von Grafiken
\usepackage[export]{adjustbox}

%
% Spezielle Schrift im Koma-Script setzen.
%
\setkomafont{sectioning}{\normalfont\bfseries}
\setkomafont{captionlabel}{\normalfont\bfseries} 
\setkomafont{pagehead}{\normalfont\bfseries} % Kopfzeilenschrift
\setkomafont{descriptionlabel}{\normalfont\bfseries}

%
% Zeilenumbruch bei Bildbeschreibungen.
%
\setcapindent{1em}
%\setlength{\abovecaptionskip}{0pt}

\usepackage{fancyhdr}
\pagestyle{fancy}
%\fancyhead{}
%\fancyfoot{}
%\fancyhead[LE,RO]{\leftmark}
%\fancyhead[LO]{\rightmark}
%\fancyhead[RE]{\thepage}
%\fancyhead[R]{\leftmark}
\fancyhead[RE]{\leftmark}
\fancyhead[LO]{\rightmark}
\fancyfoot[C]{\thepage}
\fancyhead[RO]{\includegraphics[width=75pt]{Logos/HM_Deu_CMYK_Graust}}
\fancyhead[LE]{\textsc{Diplomarbeit}}
\fancypagestyle{plain}{}
\renewcommand{\chaptermark}[1]{\markboth{\thechapter{} #1}{}}
\renewcommand{\sectionmark}[1]{\markright{\thesection{} #1}{}}
\setlength{\headheight}{37pt} 


%\usepackage{scrpage2}
%\pagestyle{scrheadings}
% Inhalt bis Section rechts und Chapter links
%\automark[section]{chapter}

% Mitte: leer
%\chead{}

%
% mathematische symbole aus dem AMS Paket.
%
\usepackage{amsmath}
\usepackage{amssymb}
\usepackage[fixamsmath, disallowspaces]{mathtools}

%
% Type 1 Fonts für bessere darstellung in PDF verwenden.
%
\usepackage{mathptmx}           % Times + passende Mathefonts
\usepackage[scaled=.92]{helvet} % skalierte Helvetica als \sfdefault
\usepackage{courier}            % Courier als \ttdefault

%
% Paket um Textteile drehen zu können
%
\usepackage{rotating}




%
%1 Glossaries / Verzeichnisse
%
\usepackage[nonumberlist,toc,nopostdot]{glossaries}
% Entferne alphabetische Gruppierung des Glossars
\renewcommand{\glsgroupskip}{}
\makeglossaries


%
% Paket um LIstings sauber zu formatieren.
%
\usepackage[savemem]{listings}
\lstloadlanguages{TeX}



%
% Neue Umgebungen
%
\newenvironment{ListChanges}%
	{\begin{list}{$\diamondsuit$}{}}%
	{\end{list}}

%
% aller Bilder werden im Unterverzeichnis figures gesucht:
%
\graphicspath{{bilder/}}


%
% Anführungsstriche mithilfe von \textss{-anzufuehrendes-}
%
\newcommand{\textss}[1]{"`#1"'}

%
% Strukturiertiefe bis subsubsection{} möglich
%
\setcounter{secnumdepth}{4}

%
% Dargestellte Strukturiertiefe im Inhaltsverzeichnis
%
\setcounter{tocdepth}{4}

%
% Zeilenabstand wird um den Faktor 1.5 verändert
%
%\renewcommand{\baselinestretch}{1.5}
\usepackage{setspace} %Zeilenabstand einstellbar



\usepackage{caption}


\DeclareUnicodeCharacter{00A0}{ }



\usepackage{verbatim}




%\newcommand{\submissiondate}{27. August 2015}
%\newcommand{\submissionmonthyear}{August 2015}


% Tabellenprogramm
\usepackage{tabularx}
\usepackage{multirow}

% Plotten von Tabellen
\usepackage[svgnames]{xcolor}
\usepackage{pgfplots}
\pgfplotsset{every axis/.append style={
thick,
tick style={thick}},
%width=10cm,
compat=1.12,
cycle list={Bihler1\\Bihler2\\green\\orange\\},
}
\usepgfplotslibrary{external}
\usetikzlibrary{pgfplots.groupplots}



\usepackage{siunitx}



\addto\captionsngerman{
\renewcommand{\figurename}{Abb.}
\renewcommand{\tablename}{Tab.}
%\renewcommand{\refname}{Quellenverzeichnis}
% \renewcommand{\bibname}{Quellenverzeichnis}
}




% Bihlerfarben
\definecolor{Bihler1}{RGB}{21, 78, 113}
\definecolor{Bihler2}{RGB}{146, 46, 46}




\usepackage[section]{placeins}

\usepackage{pdfpages}

\usepackage{lscape}


\usepackage{colortbl}






\usepackage{microtype}
%\overfullrule=2cm



\newcommand*\cleartoleftpage{%
  \clearpage
  \ifodd\value{page}\hbox{}\newpage\fi
}





\begin{comment}
%
% Paket für Links innerhalb des PDF Dokuments
%
\definecolor{LinkColor}{rgb}{0,0,0.5}
\usepackage[%
	pdftitle={Titel},% Titel der Diplomarbeit
	pdfauthor={Jens Schmelkus},% Autor(en)
	pdfcreator={LaTeX, LaTeX with hyperref and KOMA-Script},% Genutzte Programme
	pdfsubject={Diplomarbeit}, % Betreff
	pdfkeywords={Keywords}]{hyperref} % Keywords halt :-)
\hypersetup{colorlinks=false,% Definition der Links im PDF File
	linkcolor=LinkColor,%
	citecolor=LinkColor,%
	filecolor=LinkColor,%
	menucolor=LinkColor,%
	pagecolor=LinkColor,%
	urlcolor=LinkColor}
	
	
\definecolor{Colour1}{HTML}{1B9E77}
\definecolor{Colour2}{HTML}{D95F02}
\definecolor{Colour3}{HTML}{7570B3}
\definecolor{Colour4}{HTML}{E7298A}
\definecolor{Colour5}{HTML}{66A61E}
\definecolor{Colour6}{HTML}{E6AB02}
\definecolor{Colour7}{HTML}{A6761D}
\end{comment}


% Zinz Biblio-Management
% Literaturverzeichnis
% \bibliography{literatur/bib}

\begin{document}

% Römische Nummerierung für Sonderseiten, wie Verzeichnisse und Anhang
\pagenumbering{Roman}

% Titelblatt
\begin{titlepage}
\setcounter{page}{1}
%\thispagestyle {empty}
%\fancypagestyle{plain}{}
\begin{center}
%\vspace{2cm}
%\setlength{\headheight}{15pt}
\begin{figure}[h!]
\vspace{0cm}
\centering
% \includegraphics[width=0.6\textwidth]{graphics/HM_Deu_CMYK}
\includegraphics[width=0.8\textwidth]{bilder/Logos/FK03_CMYK_Block.png}
\\[0.8cm]
% \includegraphics[width=0.25\textwidth]{bilder/}
\end{figure}
\vspace{1.5cm}

{\fontsize{20}{60}\scshape Konstruktionsarbeit im vierten und Fünften Semester} 
\\[1.1cm]

\begin{doublespace}
{\fontsize{30}{22}\selectfont \textbf{Vergleichende Strukturauslegung eines Flügels in Schalen und Rippenabuweise nach CS 23}\par} 
\vspace{1.4cm}
\end{doublespace}

\title{Vergleichende Strukturauslegung eines Flügels in Schalen und Rippenabuweise nach CS 23}
\author{Jens Schmelkus}
\date{Dezember 2017}

{\fontsize{23}{60}\scshape Jens Schmelkus} 
\\[2.0cm]


\textbf{Betreuer}: Prof. Dr. Guido Sperl



% \textbf{Fakultät}: Fakultät für Maschinenbau, Fahrzeugtechnik, Flugzeugtechnik

\textbf{Studiengang}: Fahrzeugtechnik 

\textbf{Abgabe}: 20.12.2017

\vfill

% \colorbox{orange}{Name oben drüber Betreuer kleiner als meiner, Fakultät, keine Tabellenform, Diplomarbeit kleiner}

\end{center}
\end{titlepage}


\cleardoublepage\thispagestyle{empty}

%% \chapter*{Eigenständigkeitserklärung}
% \markboth{Eigenständigkeitserklärung}{Eigenständigkeitserklärung}
\chapter*{Erklärung}\addcontentsline{toc}{chapter}{Erklärung}
\markboth{Erklärung}{Erklärung}

Hiermit wird erklärt, dass die Arbeit mit obigem Thema selbständig verfasst und noch nicht anderweitig für Prüfungszwecke vorgelegt wurde. Weiterhin sind keine anderen als die angegebenen Quellen oder Hilfsmittel verwendet und wörtliche sowie sinngemäße Zitate als solche gekennzeichnet worden. \\[2cm]
\begin{tabularx}{\textwidth}{lX}
 München, den \_\_\_\_\_\_\_\_\_\_\_\_\_\_\_\_\_\_\_\_ \hspace{10 mm} & \_\_\_\_\_\_\_\_\_\_\_\_\_\_\_\_\_ \\[0.2cm]
 & \hspace{5 mm} {\footnotesize Jens Schmelkus}
\end{tabularx}





\cleardoublepage

% % Die eidesstattliche Erklärung mit Unterschrift
\chapter*{Sperrvermerk}\addcontentsline{toc}{chapter}{Sperrvermerk}

Diese Diplomarbeit enthält vertrauliche Informationen. Sie darf nicht vervielfältigt oder auf elektronischem Weg verteilt werden. Einsichtnahme ist für einen Zeitraum von 5 Jahren ab Abgabe nur für Prüfungszwecke gestattet.




\clearpage

\chapter*{Zusammenfassung}\addcontentsline{toc}{chapter}{Zusammenfassung}

In dieser Arbeit wird der Betriebsbereich für den Tragflügel eines Unbemannten Kleinstflugzeuges Ausgelegt, sowie die Strukturellen Anforderungen und Mechanischen Grenzen Formuliert. Des weiteren werden die Kräfte und Momente für die Lastfälle Betrachtet welche kritisch für die Auslegung des Struktur sind.
Es erfolgt eine Beschreibung des bisherigen Baumethoden und ein Vergleich verschiedener im Modellbau üblichen Baumethoden.
Daraus werden die Verbesserungsmöglichkeiten und Problemstellungen Formuliert.
Es wurde ein Flügel mit 2,8 m Spannweite und 0,3 m Wurzeltiefe konzipiert und mit Traglinienmethoden in XFLR5 simuliert.
Die aus der Berechnung mit XFLR gewonnen Kräfte und Momente gehen in die Auslegung der Struktur ein.
Damit wurde ein Grobmodell der Halbspannweite des Tragflügels mit CAD Methoden erstellt.


\clearpage

% Verzeichnisse
% Kopfzeile links Kapitel, rechts leer
%\ihead{\leftmark}
%\ohead{}
%\input{Sonderseiten/verzeichnisse}

%
% Inhaltsverzeichnis
%
\tableofcontents \addcontentsline{toc}{chapter}{Inhaltsverzeichnis}%\pdfbookmark[0]{Inhaltsverzeichnis}{sumario_label_pdf}


\chapter*{Nomenklatur}
%\markboth{Nomenklatur}{Nomenklatur}
\addcontentsline{toc}{chapter}{Nomenklatur}
%
% In diesem Abschnitt werden alle in diesem Dokument verwendeten Bezeichner beschrieben.

% Falls zur jeweiligen Größe kein Koordinatensystem gegeben ist, wird dieses als Index spezifiziert oder ist nicht zutreffend.

%\section*{Kleine lateinische Bezeichner}
%\vspace{0.2cm}\noindent
%\begin{tabularx}{\textwidth}{lX}
 %$a$ & Schallgeschwindigkeit\\
%\end{tabularx}
%\vspace{0.5cm}\noindent


\section*{Formelzeichen}

\begin{tabularx}{\textwidth}{lll}
\textbf{Symbol} & \textbf{Einheit} & \textbf{Beschreibung}\\
& & \\
$A$ & \si{\meter\squared} & Querschnittsfläche \\
$D$ & \si{\milli\meter} & Durchmesser \\
$f$ & \si{\hertz} & Frequenz \\
$f(z)$ & 1 & normiertes Bewegungsgesetz \\
$J$ & \si{\kilogram\meter\squared} & Massenträgheitsmoment \\
$k_\mathrm{v}$ & \si{\meter\squared} & Durchflussbeiwert  \\
$K_v$ & \si{\cubic\meter\per\hour} & Durchflussfaktor \\
$M$ & \si{\newton\meter} & Moment \\
$n$ & \si{1\per\minute} & Drehzahl \\
$p$ & \si{\milli\meter} & Spindelsteigung \\
$p$ & \si{\bar} & Druck \\
$Q$ & \si{\cubic\meter\per\hour} & Dichte \\
$s(\varphi)$ & \si{\milli\meter\per\radian} & Weg des Abtriebsgliedes (Übertragungsfunktion 0. Ordnung) \\
$S$ & \si{\milli\meter} & Gesamtweg des gerade geführten Abtriebsgliedes \\
$t$ & \si{\second} & Zeit \\
$z$ & 1 & normierter Drehwinkel eines Bewegungsabschnittes \\
$Z$ & 1 & Anzahl \\
$\zeta$ & 1 & Druckverlustbeiwert \\
$\rho$ & \si{\kilogram\per\liter} & Dichte \\
$\varphi$ & \si{\radian} & Drehwinkel des Kurvenkörpers \\
$\Phi_{ik}$ & \si{\radian} & Gesamtdrehwinkel des Kurvenkörpers im Abschnitt $ik$  \\
\end{tabularx}

\clearpage

\section*{Indizes}

\begin{tabularx}{\textwidth}{ll}
\textbf{Symbol} & \textbf{Beschreibung}\\
& \\
$A$ & Außenring \\
$G$ & Gewinderollentrieb \\
$I$ & Innenring \\
$ik$ & Nummerierung der Bewegungsabschnitte \\
$max$ & Maximal \\
$n$ & Drehzahl \\
$NCA$ & Auf das Numerical Controlled Aggregat bezogen \\
$Ring$ & Ringkontakt am Wälzkörper \\
$T$ & Teilkreis \\
$ue$ & Überrollfrequenz \\
$W$ & Wälzkörper \\
$WA$ & Wälzkörper \\
$zus$ & zusätzlich \\
\end{tabularx}



\begin{comment}

\vspace{0.2cm}\noindent
\begin{tabularx}{\textwidth}{lll}
Symbol & Einheit & Beschreibung\\
$t$ & \si{\second} & Zeit \\
$p$ & \si{\milli\meter} & Spindelsteigung \\
$f(z)$ &  & Normierte Übertragungsfunktion \\
$n$ & \si{1\per\minute} & Drehzahl des Hauptantriebs \\
$a$ & \si{1\per\second\squared} & Winkelbeschleunigung \\
$u_{max}$ & \si{1\per\minute} & Maximale Drehzahl des NCA's \\
$S_{ik}$ & \si{\milli\meter} & Gesamtweg des gerade geführten Abtriebsgliedes im Abschnitt $ik$ \\
$\omega_a = \dot{\phi}$ & \si{\radian\per\second} & Winkelgeschwindigkeit des Kurvenkörpers \\
$\Phi_{ik}$ &   & Gesamtdrehwinkel des Kurvenkörpers im Abschnitt $ik$  \\
$ik$ & \si{\radian} bzw. \si{\degree} & Nummerierung der Bewegungsabschnitte \\
$z$ & 1 & normierter Drehwinkel eines Bewegungsabschnittes \\
$\varphi$ & \si{\radian} bzw. \si{\degree} & Drehwinkel des Kurvenkörpers \\
$s(\varphi)$ & \si{\milli\meter\per\radian} & Weg des Abtriebsgliedes (Übertragungsfunktion 0. Ordnung) \\
$n_{max}$ & \si{1\per\minute} & Maximale Drehzahl der Maschine \\
\end{tabularx}



 

\colorbox{orange}{nicht Aktuell muss noch überarbeitet werden}


\end{comment}




%  $u_{max}$ in $\frac{\text{U}}{min}$
 
 
 % TODO: Indizes


%
% Abkürzungsverzeichnis
%
%


\newacronym[plural=NCAs,firstplural=Numerical  Controlled  Aggregate]{NCA}{NCA}{Numerical Controlled Aggregat}
\newacronym{VC1}{VC 1}{Vari Control 1}
\newacronym{FFT}{FFT}{Fast Fourier Transformation}
\newacronym{HFFT}{HFFT}{Hüllkurven Fast Fourier Transformation}
\newacronym{NC}{NC}{Numerical Controlled}
\newacronym{CNC}{CNC}{Computerized Numerical Controlled}
\newacronym[plural=cM,firstplural=centiMorgans (cM)]{cM}{cM}{centiMorgan}
\newacronym{SPS}{SPS}{speicherprogrammierbare Steuerung}
\printglossary[title={Abkürzungsverzeichnis},toctitle={Abkürzungsverzeichnis},type=\acronymtype,style=long]


% Merke mir die römische Seitenzahl in 'roemisch' und setzte Nummeriernung 

% auf arabisch für die eigentlichen Kapitel
\cleardoublepage

\newcounter{roemisch}
\setcounter{roemisch}{\value{page}}
\pagenumbering{arabic}

\chapter{Ausgangslage und Zielstellung}\label{cha:Ausgangslage und Zielstellung}

Die in diesem Bericht dokumentierte Auslegung beschäftigt sich mit dem Tragflügel des studentischen AUVSI Flugzeugs, welches im Labor für Systemtechnik entwickelt wurde. Die vorangegangenen Flugzeuggenerationen und das Reglement des Wettbewerbs bilden die Ausgangslage für die Weiterentwicklung.

\section{Historie des AUVSI SUAS-Wettbewerbs}
Seit 2002 findet jährlich der AUVSI (Association for Unmanned Vehicle Systems International) SUAS (Student Unmanned Aerial Systems Competition)-Wettbewerb in den USA statt.

Hieran hat das studentische Hochschulteam in den Jahren 2015 und 2016 erfolgreich teilgenommen. Bereits im Jahr 2014 gab es die ersten Anstrengungen eine für diesen Wettbewerb passende Entwicklung einer Flugplattform. Leider war dies zum gegebenen Zeitpunkt noch nicht zum Erfolg. Detaillierte Informationen hierzu können der Diplomarbeit von Herrn Dipl. Ing. Fabian Meilinger entnommen werden \cite{Meiling}.

Die meisten Anforderungen an die Flugplattform ergeben sich aus dem Reglement des genannten studentischen Wettbewerbs. 

\clearpage

\section{Das Flugzeug des AUVSI SUAS Teams}

Aufbauend auf den ersten Erfahrungen in der Saison 2014 folgt der konzeptionelle Aufbau des Flugzeugs einem modularen Konzept. Dies ermöglichte mit definieren Schnittstellen zwischen den Baugruppen eine getrennte Weiterentwicklung der jeweiligen Einzelkomponenten. Hieraus entstanden zum Beispiel 2015 Variationen für die Forschungsmission im Regenwald Ecuadors, bei der eine deutlich voluminösere Nutzlast mitgeführt werden sollte. \cite{Niclas}.

\begin{figure}[H]
\centering
\includegraphics[width=0.9\textwidth]{bilder/Fotos/AUVSI_2015.jpg} 
\caption{Das AUVSI 2015 Modell im Einsatzzustand am Testflugplatz} 
\label{fig:Das AUVSI 2015 Modell in Einsatzzustand am Testflugplatz}
\end{figure}

Im Bild \ref{fig:Das AUVSI 2015 Modell in Einsatzzustand am Testflugplatz} ist der Gesamtaufbau des Flugzeugs zu sehen. Bisher sind der Einsatz von rechteckigen Tragflächen und Leitwerken sowie CFK-Rohren als Ausleger für Heck und Motorträger charakteristisch. Unter dem zentralen Rumpfrohr ist der zylinderförmige Nutzlastbehälter montiert.


\clearpage


\section{Bisherige Flügelbauweise}

Die bisherigen Tragflügel sowie einige Leitwerksflächen wurden in der, in Modellbaukreisen, als \glqq Styro-Abachi\grqq{} bekannte Bauweise ausgeführt. Die Tragflächen des AUVSI-UAV's sind bisher von einem Kern aus Polystyrol gefertigt worden. In einem ersten Schritt wird mit Hilfe einer 4-Achs CNC Heißdrahtschneidemaschine die gewählte Profilform (Clark-Y) aus einem Styroporkern geschnitten. In diesen Rohling werden anschließend Aussparungen für die Kabelschächte und die Servomotoren der Rudermaschinen eingebracht. Auf Ober- und Unterseite der Fläche wird Abachi Furnier mit einer Schichtstärke von 0,6 mm verklebt. Dabei wird während der Aushärtephase des Klebstoffs ein Vakuumsack eingesetzt, um die Festigkeit der Klebeverbindung zu steigern.
Auf beide Enden werden Endrippen aus 3 mm starkem Ceibasperrholz aufgebracht. Die Kraftübertragung in die anschließenden Segmente wird durch, in die Polystyrolblöcke eingeklebte, CFK-Rohre realisiert. Daran anschließend erfolgt das Auskasten und Anschlagen der Ruderflächen.
Abschließend wird die Fläche mit heißklebender Kunststofffolie mit Hilfe eines Bügeleisens überzogen zur Steigerung der Resistenz  gegen Umwelteinflüsse.

Bei dieser Bauweise soll der Großteil der Zugspannungen vom aufgebrachten Abachi Furnier in Faserrichtung aufgenommen werden. Der Polystyrolkern soll hingegen die Übertragung der Schubspannungen zwischen Ober- und Unterseite der Tragfläche gewährleisten und nimmt zusätzlich entstehenden Druck quer zur Oberfläche auf. Als problematischer Parameter, der die Biegefestigkeit der Gesamtstruktur erheblich reduziert, hat sich die Festigkeit der Klebeverbindung zwischen oberer Furnierlage (Druck) und dem Polystyrolkern herauskristallisiert. Ein Versagen der Verklebung führt zum Abschälen des Furniers, was Druckbeulen der oberseitigen Furnierlage nach sich zieht. Dies führt zu einem Stabilitätsversagen.

Insgesamt hat sich diese Bauweise als sehr robust erwiesen, sie  hat eine ausreichende Stabilität in allen Flugsituationen nachgewiesen.
Die Formabweichungen von der gewählten Profilform sind vergleichsweise gering. Dies setzt allerdings den Einsatz von korrekt gefertigten Werkzeugen während des Vakuumverfahrens voraus und eine fehlerfreie Durchführung der oben genannten Verfahrensschritte.  

Nachteilig ist das vergleichsweise hohe Gewicht von \SI[per-mode=symbol]{2,145}{\kilogram\per\meter\squared} [AUVSI/Ecuador 2015].
Zusätzlich wird für die Fertigung ein einfacher Werkzeugbau benötigt und es ergeben sich zahlreiche Arbeitsschritte, die zum Teil hohes handwerkliches Geschick erfordern. Damit variiert die Qualität der Ergebnisse deutlich je nach Erfahrung und Sorgfalt des fertigenden Personals. Besonders Fehler beim Vakuumverkleben der Furnierschichten können das gesamte Tragflächensegment unbrauchbar machen. 

Bisher sind weder eine Vorauslegung noch eine Dimensionierung der Struktur für die erwarteten Lasten durchgeführt worden.

Erfahrungen aus Belastungs- und Überlastungsversuchen mit früheren Tragflächen dieser Bauweise lassen erwarten, dass die bisherige Struktur bedeutend überdimensioniert ist mit negativen Auswirkungen auf die erzielbaren Flugleistungen. 

\clearpage

\section{Verbesserungsziele}

Ziele dieser Arbeit sind Verbesserungen der jetzigen Bauform in folgenden Bereichen:
\begin{itemize}
    \item Nachweis der Festigkeit der Fläche bei bekannten Sicherheiten und Lastgrenzen
    \item Reduzierung des Flächengewichts durch bessere Ausnutzung der Materialkennwerte
    \item Reduzierung des Anteils an anspruchsvoller Handarbeit und Steigerung des Anteils mechanisierter und werkzeugbasierter Fertigung
    \item Einsatz einer möglichst profiltreuen Bauweise bis zu mindestens 25\% der Flächentiefe.
    \label{lst:Verbesserungsziele}
\end{itemize}


\section{Optimierungsgrenzen durch den Einsatz}

Der Optimierung des Flügels auf die in \ref{lst:Verbesserungsziele} genannten Ziele sind durch den praktischen Einsatz einige Grenzen gesetzt.
Im Folgenden sollen Situationen beschrieben werden, die im Alltagseinsatz des Fluggeräts auftreten und die zu Belastungen der Struktur führen, die weit über die im statischen Flugzustand  liegen können. 

Mit dem Fluggerät wird im Normalfall von einer kurzen Rampe ein Flitschenstart durchgeführt. Dabei wird das UAV durch ein Gummiseil bis knapp über die festgelegt Mindestfluggeschwindigkeit beschleunigt. Dadurch entstehen kurzzeitige Lastvielfache von bis zu 3,5 g in Flugrichtung.
Auch ein Hochstart wurde für spezielle Einsätze in Erwägung gezogen, in denen der Energieverbrauch zum Erreichen der Missionshöhe einen zu hohen Anteil an der Gesamtflugaufgabe hätte. 
Dabei muss die Tragfläche kurzzeitig deutlich höhere Belastungen durch  Auftriebskräfte als beim statischen Reiseflug aushalten. 

Zusätzlich treten im Alltagseinsatz eine Reihe von atypischen Belastungen auf. So werden die Tragflächen beispielsweise in Transportkisten befördert, beim Aus- und Einladen sowie bei der Montage durch Handkräfte werden diese auf Druck beansprucht. Deshalb bedarf es entweder einer ausreichenden Druckfestigkeit oder einfacher Reparaturmethoden, um die entstandenen Beschädigungen durch unsachgemäße Handhabung zu beseitigen. 


\cleardoublepage

\chapter{Auswahl der Bauweise}


In diesem Kapitel werden verschiedene Strukturbauweisen verglichen und ihre Tauglichkeit für den vorgesehen Einsatz bewertet.

\section{Betrachtung bisheriger Bauformen}
\label{cha:Statistische Betrachtung bisheriger Bauformen}

Um eine Auswahl der Bauweise nach Gewichtsgesichtspunkten zu treffen werden in der Tabelle \ref{tab:Flächengewichte Für Modelltragflügel} verschiedene realisierte Bauweisen einer Flügelstruktur von Faserverbund bis Folienbespannter Rippenbauweise aufgelistet. Dabei wird zur Vergleichbarkeit das Strukturgewicht je Projezierter Fläche errechnet.

Unberücksichtigt bleibt hier die jeweilig Maximal ertragbare Last jeder einzelnen Ausführung wodurch lediglich ein Vergleich der realisierbaren Bauarten erfolgen kann.
\begin{table}
\centering
\begin{adjustbox}{angle=90}
\begin{tabular}{|l|l|l|l|l|l|l|l|l|l|l|l|}
\hline
Name des Fluegels              & Hersteller         & Spannweite [mm] & Flügel MAC [mm] & Wurzeltiefe [mm] & Flaeche [m²] & Gewicht [g] & Gewicht [kg] & Strukturgewicht [kg/m²] & Bauweise \\ \hline
Ultraleicht Hangsegler         & Benjamin Bachmaier & 641             & 170              & 170              & 0,109         & 42,2        & 0,0422       & 0,387                    & Balsarippen Folienbespannt Kieferholm \\\hline
Ultraleicht Motorflieger       & Jens Schmelkus     & 1000            & 192              & 192              & 0,192         & 83,5        & 0,0835       & 0,435                    & Kiefer Sperrholzrippen auf CFK-Rohren Folienbespannt  \\\hline
Moravia Parkflyer              & Moravia            & 1190            & 200              & 200              & 0,238         & 195,7       & 0,1957       & 0,822                    & Balsarippen, Balsa Doppel T-Träger Folienbespannt         \\\hline
Kartonflügel Masterarbeit     & Ingrid             & 1450            & 272              & 295              & 0,394         & 380         & 0,3800       & 0,963                    & Mehrschicht Wellkarton Folienbespannt                            \\\hline
Kartonflügel Testsample       & Ingrid             & 200             & 245              & 245              & 0,049         & 47,8        & 0,0478       & 0,976                    & Mehrschicht Wellkarton Folienbespannt                          \\\hline
AUVSI RippensegmentTestversion & Jens Schmelkus     & 700             & 300              & 300              & 0,210         & 251,5       & 0,2515       & 1,198                    & Ceiba Rippen mit CFK Rohrholm Folienbespannt                     \\\hline
Easy Star                      & Multiplex          & 720             & 200              & 200              & 0,144         & 211,7       & 0,2117       & 1,470                    & Geschäumtes Polystyrol mit Glasfaserohrholm                     \\\hline
AUVSI 2016                     & Fa. Gewalt         & 1495            & 250              & 250              & 0,374         & 745         & 0,7450       & 1,993                    & Polystyrolkern mit Abachifunier Beplankt                         \\\hline
Ecuador 2015                   & Helmut Birzer      & 695             & 252              & 252              & 0,175         & 375,7       & 0,3757       & 2,145                    & Polystyrolkern mit Abachifunier Beplankt                         \\\hline
Chernobyl 2016                 & Team AUVSI         & 690             & 370              & 370              & 0,255         & 633,1       & 0,6331       & 2,480                    & Polystyrolkern mit Abachifunier Beplankt CFK Rohr                \\\hline
DS-Carbonfläche               & Steffen Bieg       & 1500            & 208              & 245              & 0,290         & 852         & 0,8520       & 2,938                    & Kohlefaser SchalenbauweiseIn Kohlefaserform Negativ \\  \hline               
\end{tabular}
\end{adjustbox}
\caption{Flächengewichte Für Modelltragflügel}
\label{tab:Flächengewichte Für Modelltragflügel}
\end{table}

\section{Wahl der Holmform}

Um die gestellte Flugaufgabe mit der notwendingen Strukturellen Sicherheit zu erfüllen soll die Bauweise eine einfache Skalierbarkeit ihrer Tragenden Elemente ermöglichen.
Die verfügbaren Geometrien eignen sich bezüglich ihrer Verhältnisse zwischen Widerstandsmoment gegen Biegung sowie Torsion und Querschnittsfläche unterschiedlich gut.

Zunächst werden aus den bereits in \ref{Resultierende Kräfte und Momenten Stationär} sowie in \ref{Resultierende Kräfte und Momenten Kritisch} gewonnenen Belastungen in Zusammenhang mit dem nach \ref{} zur Verfügung stehen Bauraum in notwendige Trägheitsmomente umformuliert.

Für die Festigkeitsauslegung wird nach \ref{} ein Sicherheitsfaktor von $ J = 1,5 $ gewählt.

\section{Nötiger Fertigungsaufwand}

Zur Herstellung der Verschiedenen Bauweise ist ein zum Teil erheblich unterschiedlicher Produktionsaufwand von nöten. Insbesondere die Herstellung von Vorrichtungen und Formen unterscheidet sich zwischen den Methoden. 

\cleardoublepage

\chapter{Bestimmung der Lasten nach CS 23}\label{cha:Bestimmung der Lasten nach CS 23}

\section{Ausgangsannahmen für das Fluggerät}

Das Fluggerät und der Tragflügel werden für die gegebene Flugaufgabe anhand folgender Parameter ausgelegt:

\begin{table}[h]
\centering
\begin{tabular}{|l|l|l|l|}
\hline
Parameter  & Bezeichnung &  Wert & Einheit \\ \hline
Fluggeschwindigkeit  & $V_{reise}$ & 15,0 & $m/s$\\ \hline
Abfluggewicht & $m_{reise}$  & 5,0 & $kg$\\ \hline
Profilierung & Clark-Y & - & - \\ \hline
Segmentteilung & $Y_{seg}$ & 700 & mm\\ \hline
\end{tabular}
\caption{Grundannahmen für das System}
\label{tab:Grundannahmen für das System}
\end{table}

\subsection{Sicherheiten und Anforderungen}

Bei der Berechnung der Lasten und der drauf aufbauenden Dimensionierung wird von den in Tabelle \ref{} im Anhang dargelegten Sicherheiten und Anforderungen an das Flugzeug  ausgegangen.

\subsection{Der Tragflügel}

Der Tragflügel wird für die gegebene Flugaufgabe anhand folgender Parameter ausgelegt:

\begin{figure}[H]
\centering
\includegraphics[width=0.9\textwidth, trim={15mm 40mm 15mm 40mm},clip]{bilder/Fotos/Grundriss_Flaeche.pdf}
\caption{Aufriss mit Eckdaten der Fläche} 
\label{fig:Flächenaufriss}
\end{figure}

Die Tragfläche setzt sich aus einem zentralen rechteckigen Segment und jeweils daran angeschlossenen Außensegmenten 
in Trapezform zusammen.
Hierbei werden Eigenschaften von bisher eingesetzten Flügeln übernommen.
Eine weitergehende Diskussion über die Wahl dieser aerodynamischen Parameter soll in dieser Arbeit nicht stattfinden. 
%\colorbox{red}{Es wird von der Strukturauslegung ausgegangen.}
Die Wahl von geradlinig aufgebauten Tragflächensegmenten ermöglicht hier Vorteile bei der Fertigung.\\
Im Bild \ref{fig:Flächenaufriss} ist ein Aufriss mit den Grundabmessungen dargestellt,welche für die aerodynamische Berechnung und den mechanischen Aufbau verwendet werden.

\begin{table}[h]
\centering
\begin{tabular}{|l|l|l|l|}
\hline
Parameter  & Bezeichnung &  Wert & Einheit \\ \hline
Spannweite  & $V_{reise}$ & 2800 & $mm$\\ \hline
Projizierte Fläche & $A_{ref}$  & 0,752 & $m^2$\\ \hline
Wurzeltiefe & $CH_{Wurzel}$ & 300 & $mm$ \\ \hline
Spitzenverwindung & $\alpha_{Spitze}$ & -\,2 & $^\circ$\\ \hline
Mittlere Flügeltiefe & $l_{\mu}$ & 275 & $mm$ \\ \hline
\end{tabular}
\caption{Dimensionen des Tragflügels}
\label{tab:Dimensionen des Tragflügels}
\end{table}

\clearpage


\section{Bestimmung der Lasten}
Mit einem Gewicht von 5,0 kg gehört das Fluggerät gemäß CS 23 der Kategorie \glqq Normal\grqq{} an.
Alle Auslegungsrechnungen beziehen sich auf diese Eingliederung.


\subsection{Lasten im stationären Flug}
Im Folgenden zeigt die Abbildung \ref{fig:Stationärer Reiseflug in XFLR} den aeroynamischen Entwurf im stationär ausgelegten Reiseflug.

\begin{figure}[H]
\centering
\includegraphics[width=0.9\textwidth]{bilder/Fotos/Aeroentwurf_Fluegelkonstruktion.png}
\caption{Stationärer Reiseflug bei 15 $m/s$ - entnommen den Berechnung in XFLR} 
\label{fig:Stationärer Reiseflug in XFLR}
\end{figure}

In diesem Fall ergibt die Ermittlung der aerodynamischen Beiwerte folgende Situation:

\begin{table}[h]
\centering
\begin{tabular}{|l|l|l|l|}
\hline
Parameter  & Bezeichnung &  Wert & Einheit \\ \hline
Fluggeschwindigkeit  & $V_{reise}$ & 15,00 & $m/s$\\ \hline
Abfluggewicht & $m_{reise}$  & 5,0 & $kg$\\ \hline
Anstellwinkel & $\alpha$ & 2,45 & $^\circ$ \\ \hline
Auftriebsbeiwert & $C_{L}$ & 0,473 & -\\ \hline
Widerstandsbeiwert & $C_{D}$ & 0,016  & -  \\ \hline
\end{tabular}
\caption{Errechnete Werte im stationären Reiseflug}
\label{tab:Errechnete Werte im Stationären Reiseflug}
\end{table}

In diesem Zustand ist der Flügel frei von Nickmomenten und das Höhenleitwerk erzeugt keine aerodynamischen Auftriebskräfte.

\clearpage

\subsection{Resultierende Kräfte und Momente}
\label{Resultierende Kräfte und Momenten Stationär}

In der stationären Flugsituation entstehen die folgenden maximalen Kräfte und Momente.

Biegemoment auf der 25\% Tiefenlinie der Tragfläche um die Einspannung in $Y = 0$
\begin{equation}
M_{b-X\,max} = 13,92 Nm    
\end{equation}

Drehmoment  um die 25\% Tiefenlinie an der Einspannung $Y = 0$
\begin{equation}
M_{t-Y\,max} \approx 0 Nm
\end{equation}
Für diesen Auslegungszustand ist der Schwerpunkt des Systems so gewähl, dass es im Arbeitspunkt nahezu momentenfrei fliegt. Das Höhenleitwerk kommt lediglich bei Abweichungen aus diesem Zustand zum Wirken.

\subsection{Kritische Lasten}

Gemäß CS 23.337 werden die positiven und negativen maximalen Manöverlastvielfachen berechnet.

Abflugmasse (take off weight) des Flugzeugs:
\begin{equation}
\label{eq:K1}
W_{0} = \SI{5}{\kilogram} = \SI{11,0231}{lb}
\end{equation}

Positive limit manoevering load factor:
\begin{equation}
\label{eq:K2}
n_{pos} = 2,1 + \frac{24000}{W_{0}+10000} = 4,4974 \qquad n_{pos gewählt}=4,5
\end{equation}

Negative limit manoevering load factor:
\begin{equation}
\label{eq:K3}
n_{neg} = (-0,4) \cdot n_{pos} = -1,7989 \qquad n_{neg_gewählt}=-1,8
\end{equation}

%/\begin{figure}[H]
%/\centering
%/\includegraphics[width=0.9\textwidth]{bilder/Formeln/Kritische_Lastfaelle_Manoeverlasten.png}
%/\caption{Limit manoevring load factors nach CS23.337} 
%/\label{fig:Limit manoevring load factors nach CS23.337}
%/\end{figure}

%/Hochstart

%/Hier soll in konservativer Auslegung der Lasten für den gegebenfalls belastendensten Fall beim Hochstart mittels Gummiseil abgeschätzt werden. Hier übersteigt die Auftriebsbelastung für den Hochanstellwinkelbereich kurzeitig den Maximalauftrieb des Profils durch den Betrieb im instationären Bereich./%

%\\subsection{\colorbox{red}{Resultierende Kräfte und Momente}}
\label{Resultierende Kräfte und Momenten Kritisch}

\clearpage

\section{Betriebsgrenzen des Tragflügels}

Im Einsatz unterliegt das Tragwerk sowohl aerodynamischen als auch mechanischen Grenzen.
Diese werden Allgemein nach \cite{Sperl2011} berechnet.

Zunächst werden die kritischen Geschwindigkeiten für den Einsatz mit und ohne Klappen bestimmt:

\begin{equation}
\label{eq:K4}
c_{\text{L\ max\ Flap}} = 1.5 \qquad c_{\text{L\ max\ Clean}} = 1.3 
\end{equation}

\begin{equation}
\label{eq:K5}
V_{\text{S\,0}} = \sqrt{\cfrac{W_{0\ \text{MTOW}} \cdot \SI[per-mode = fraction]{9,81}{\meter\squared\per\second\squared}}{c_{\text{L\ max\ Flap}} \cdot \cfrac{\rho_{\text{Luft\ 0m}}}{2} \cdot S_{Wing}}}=\SI[per-mode = fraction]{8,42581}{\meter\per\second} \qquad V_{\text{S\,0\ gewählt}}=\SI[per-mode = fraction]{8,5}{\meter\per\second}
\end{equation}

\begin{equation}
\label{eq:K6}
V_{\text{stall\ gewählt}} = \sqrt{\cfrac{W_{0\ \text{MTOW}} \cdot \SI[per-mode = fraction]{9,81}{\meter\squared\per\second\squared}}{c_{\text{L\ max\ Clean}} \cdot \cfrac{\rho_{\text{Luft\ 0m}}}{2} \cdot S_{Wing}}}=\SI[per-mode = fraction]{9,05078}{\meter\per\second} \qquad V_{\text{stall\ gewählt}}=\SI[per-mode = fraction]{9,1}{\meter\per\second}
\end{equation}

%/\begin{figure}[H]
%/\centering
%/\includegraphics[width=0.9\textwidth]{bilder/Formeln/Mindestbetriebsgeschwindikeit.png}
%/\caption{Kritische Fluggeschwindikeiten} 
%/\label{fig:Kritische Fluggeschwindikeiten}
%/\end{figure}

Im selben Verfahren wird die Abrissgeschwindikeit (stall speed) für negative Werte des Auftriebs bestimmt:

\begin{equation}
\label{eq:K7}
c_{\text{L\ min\ Clean}} = 0.8 
\end{equation}
\begin{equation}
\label{eq:K8}
V_{\text{S\,neg}} = \sqrt{\cfrac{W_{0\ \text{MTOW}} \cdot \SI[per-mode = fraction]{9,81}{\meter\squared\per\second\squared}}{c_{\text{L\ min\ Clean}} \cdot \cfrac{\rho_{\text{Luft\ 0m}}}{2} \cdot S_{Wing}}}=\SI[per-mode = fraction]{11,53752}{\meter\per\second} \qquad V_{\text{S\,neg\ gewählt}}=\SI[per-mode = fraction]{11,6}{\meter\per\second}
\end{equation}

%/\begin{figure}[H]
%/\centering
%/\includegraphics[width=0.9\textwidth]{bilder/Formeln/VS_neg.png}
%/\caption{Mindestgeschwindikeit für Invertierten Flug} 
%/\label{fig:Mindestgeschwindikeit für Invertierten Flug}
%/\end{figure}

Aus den Parametern ergibt sich im Weiteren die mindestens geforderte Höchstgeschwindigkeit beim Klappeneinsatz:

\begin{equation}
\label{eq:K8}
1.4 \cdot V_{\text{S\,0\ gewählt}}=\SI[per-mode = fraction]{11,9}{\meter\per\second} \quad
1.8 \cdot V_{\text{S\,0\ gewählt}}=\SI[per-mode = fraction]{16,38}{\meter\per\second} \quad
V_{\text{F\ gewählt}}=\SI[per-mode = fraction]{18}{\meter\per\second}
\end{equation}

%/\begin{figure}[H]
%/\centering
%/\includegraphics[width=0.9\textwidth]{bilder/Formeln/VF.png}
%/\caption{Klappenhöchstgeschwindigkeit} %/\label{fig:Klappenhöchstgeschwindigkeit}
%/\end{figure}

Des Weiteren ergeben sich die Manövergeschwindigkeiten $V_{A}$ und $V_{G}$ nach CS 23.335 zu:

\begin{equation}
\label{eq:K8}
V_{\text{A}}=V_{\text{S\,1\ gewählt}} \cdot \sqrt{n_{\text{pos}}}=\SI[per-mode = fraction]{19,304}{\meter\per\second} 
\end{equation}
\begin{equation}
\label{eq:K8}
V_{\text{G}}=V_{\text{S\,neg\ gewählt}} \cdot \sqrt{n_{\text{neg}}}=\SI[per-mode = fraction]{15,563}{\meter\per\second} 
\end{equation}

%/\begin{figure}[H]
%/\centering
%/\includegraphics[width=0.9\textwidth]{bilder/Formeln/V_g.png}
%/\caption{Manövergeschwindigkeiten} 
%/\label{fig:Manövergeschwindigkeiten}
%/\end{figure}

\clearpage

Für den Dauerarbeitspunkt wird die Design Cruise Speed $V_{C}$ bestimmt:

\begin{equation}
\label{eq:K8}
W=\SI{11}{lb} \quad S=\SI{8,1}{ft^2}
\end{equation}
\begin{equation}
\label{eq:K8}
V_{\text{C\,min}}=33 \cdot \sqrt{\cfrac{W}{S}}=\SI{38,4563}{kn}=\SI[per-mode = fraction]{19,7804}{\meter\per\second}
\end{equation}
Berechnung von $V_{\text{C\,max}}$ entfällt mangels einer Leistungsauslegung
\begin{equation}
\label{eq:K8}
V_{\text{C\ gewählt}}=\SI[per-mode = fraction]{20}{\meter\per\second} 
\end{equation}

%/\begin{figure}[H]
%/\centering
%/\includegraphics[width=0.9\textwidth]{bilder/Formeln/V_c.png}
%/\caption{Design Cruise Speed} 
%/\label{fig:Design Cruise Speed}
%/\end{figure}

Aus dieser lässt sich die maximal nicht zu überschreitende Geschwindigkeit $V_{D}$ ableiten:

\begin{equation}
\label{eq:K8}
V_{\text{D\,min\,1}}=1.25 \cdot V_{\text{C\,max}} \quad 
\end{equation}
Berechnung von $V_{\text{D\,min\,1}}$ entfällt mangels $V_{\text{C\,max}}$
\begin{equation}
\label{eq:K8}
V_{\text{D\,min\,2}}=1.35 \cdot V_{\text{C\ gewählt}}=\SI[per-mode = fraction]{27}{\meter\per\second} 
\end{equation}

%/\begin{figure}[H]
%/\centering
%/\includegraphics[width=0.9\textwidth]{bilder/Formeln/V_d.png}
%/\caption{Maximale Sturzgeschwindikeit} 
%/\label{fig:Maximale Sturzgeschwindikeit}
%/\end{figure}

\begin{landscape}
\begin{figure}[H]
\centering
\includegraphics[height=0.945\textheight, trim={20mm 24mm 22mm 22mm},clip]{bilder/Formeln/VN-Diaggramm-rev-00.pdf}
\caption{Betriebsbereich im V\textsubscript{n}-Diagramm} 
\label{fig:Vn Diagramm des Flugsystems}
\end{figure}
\end{landscape}

\cleardoublepage

\chapter{Material und Bauartbedingte Grenzen der Optimierung}\label{cha:Material und Bauartbedingte Grenzen der Optimierung}


\cleardoublepage

%/\chapter{\colorbox{red}{Nachweis der Struktur}}\label{cha:Berechnung der Struktur}

%/\section{\colorbox{red}{Aufteilung der Lasten nach CS}}

%/\subsection{\colorbox{red}{Biegelasten}}

%/\subsection{\colorbox{red}{Querkräfte}}

%/\subsection{\colorbox{red}{Torsionsmomente}}


\cleardoublepage

\chapter{Konstruktion des Vergleichssegments}\label{cha:Konstruktion des Vergleichssegments}

\cleardoublepage

\chapter{Zeichnungssatz des Vergleichssegments}\label{cha:Zeichnungssatz des Vergleichssegments}

\cleardoublepage

\chapter{Bewertung der Ergebnisse}\label{cha:Bewertung der Ergebnisse}





\cleardoublepage

\newpage
%\pagenumbering{Roman}
%\setcounter{page}{\value{roemisch}}

%Biblio-Management
\bibliography{literatur/bib}

%
% Abbildungsverzeichnis
%
\listoffigures

%
% Tabellenverzeichnis
%
\listoftables

\cleardoublepage


% Appendix, falls vorhanden
\appendix

%\chapter{Anhang}

\begin{table}[H]
\centering
\includegraphics[width=0.9\textwidth]{bilder/Tabellen/Sicherheitsfaktoren.pdf}
\caption{Formblatt F021 - Nachweis der geforderten Sicherheit am Schnitt} 
\label{tab:Sicherheiten}
\end{table}

\clearpage

\begin{table}[H]
\centering
\includegraphics[width=0.9\textwidth]{bilder/Tabellen/MPP_Konstruktion_1.pdf}
\end{table}

\begin{table}[H]
\centering
\includegraphics[width=0.9\textwidth]{bilder/Tabellen/MPP_Konstruktion_2.pdf}
\end{table}

\begin{table}[H]
\centering
\includegraphics[width=0.9\textwidth]{bilder/Tabellen/MPP_Konstruktion_3.pdf}
\end{table}

\begin{table}[H]
\centering
\includegraphics[width=0.9\textwidth]{bilder/Tabellen/MPP_Konstruktion_4.pdf}
\end{table}

\begin{table}[H]
\centering
\includegraphics[width=0.9\textwidth]{bilder/Tabellen/MPP_Konstruktion_5.pdf}
\end{table}

\begin{table}[H]
\centering
\includegraphics[width=0.9\textwidth]{bilder/Tabellen/MPP_Konstruktion_6.pdf}
\end{table}

\begin{table}[H]
\centering
\includegraphics[width=0.9\textwidth]{bilder/Tabellen/MPP_Konstruktion_7.pdf}
\end{table}

\begin{table}[H]
\centering
\includegraphics[width=0.9\textwidth]{bilder/Tabellen/MPP_Konstruktion_8.pdf}
\end{table}

\begin{table}[H]
\centering
\includegraphics[width=0.9\textwidth]{bilder/Tabellen/MPP_Konstruktion_9.pdf}
\end{table}

\begin{table}[H]
\centering
\includegraphics[width=0.9\textwidth]{bilder/Tabellen/MPP_Konstruktion_10.pdf}
\end{table}

\begin{table}[H]
\centering
\includegraphics[width=0.9\textwidth]{bilder/Tabellen/MPP_Konstruktion_11.pdf}
\caption{Musterprüfplan} 
\label{tab:Musterprüfplan}
\end{table}

%\clearpage

\section{}\label{}




\end{document}