\documentclass[%
	pdftex,%              PDFTex verwenden
	a4paper,%             A4 Papier
	twoside,%             Einseitig/ wzeiseitig
	bibtotoc,%    		Literaturverzeichnis einfügen bibtotocnumbered: nummeriert
	liststotoc,%		Verzeichnisse einbinden in toc
	idxtotoc,%            Index ins Verzeichnis einfügen
	halfparskip,%        Europäischer Satz mit abstand zwischen Absätzen
	%chapterprefix,%       Kapitel anschreiben als Kapitel
	headsepline,%         Linie nach Kopfzeile
	footsepline,%         Linie vor Fusszeile
	pointlessnumbers,%     Nummern ohne abschließenden Punkt
	12pt,%                 Grössere Schrift, besser lesbar am bildschrim
	bibliography=totoc,
	cleardoublepage=empty,
	index=totoc,
	listof=totoc,
]{scrreprt}

%Maxi Biblio-Management
\usepackage[square,numbers]{natbib}
\bibliographystyle{unsrtdin}

%Zindath Biliothek Management
%\usepackage[backend=bibtex,bibencoding=ascii,style=numeric,citestyle=numeric-comp,defernumbers=true]{biblatex} 

%
% Seitenränder
%
\usepackage{geometry}
\geometry{a4paper, top=30mm, left=30mm, right=20mm, bottom=25mm, headsep=10mm, footskip=10mm}



%
% Paket für Übersetzungen ins Deutsche
%
\usepackage[french,english,ngerman]{babel}

%
% Pakete um Latin1 Zeichnensätze verwenden zu können und die dazu
% passenden Schriften.
%
%\usepackage[latin1]{inputenc}
% UTF8 Kompatabilität
\usepackage[utf8]{inputenc}
\usepackage[T1]{fontenc}

%
% Paket für Quotes
%
\usepackage[babel,french=guillemets,german=quotes]{csquotes}

%
% Paket zum Erweitern der Tabelleneigenschaften
%
\usepackage{array}


%
% Paket für schönere Tabellen
%
\usepackage{booktabs}

%
% Paket um Grafiken einbetten zu können
%
\usepackage{graphicx}
\usepackage{subfigure}
\usepackage{float} % zum abstellen der float umgebung von Grafiken
\usepackage[export]{adjustbox}

%
% Spezielle Schrift im Koma-Script setzen.
%
\setkomafont{sectioning}{\normalfont\bfseries}
\setkomafont{captionlabel}{\normalfont\bfseries} 
\setkomafont{pagehead}{\normalfont\bfseries} % Kopfzeilenschrift
\setkomafont{descriptionlabel}{\normalfont\bfseries}

%
% Zeilenumbruch bei Bildbeschreibungen.
%
\setcapindent{1em}
%\setlength{\abovecaptionskip}{0pt}

\usepackage{fancyhdr}
\pagestyle{fancy}
%\fancyhead{}
%\fancyfoot{}
%\fancyhead[LE,RO]{\leftmark}
%\fancyhead[LO]{\rightmark}
%\fancyhead[RE]{\thepage}
%\fancyhead[R]{\leftmark}
\fancyhead[RE]{\leftmark}
\fancyhead[LO]{\rightmark}
\fancyfoot[C]{\thepage}
\fancyhead[RO]{\includegraphics[width=75pt]{Logos/HM_Deu_CMYK_Graust}}
\fancyhead[LE]{\textsc{Diplomarbeit}}
\fancypagestyle{plain}{}
\renewcommand{\chaptermark}[1]{\markboth{\thechapter{} #1}{}}
\renewcommand{\sectionmark}[1]{\markright{\thesection{} #1}{}}
\setlength{\headheight}{37pt} 


%\usepackage{scrpage2}
%\pagestyle{scrheadings}
% Inhalt bis Section rechts und Chapter links
%\automark[section]{chapter}

% Mitte: leer
%\chead{}

%
% mathematische symbole aus dem AMS Paket.
%
\usepackage{amsmath}
\usepackage{amssymb}
\usepackage[fixamsmath, disallowspaces]{mathtools}

%
% Type 1 Fonts für bessere darstellung in PDF verwenden.
%
\usepackage{mathptmx}           % Times + passende Mathefonts
\usepackage[scaled=.92]{helvet} % skalierte Helvetica als \sfdefault
\usepackage{courier}            % Courier als \ttdefault

%
% Paket um Textteile drehen zu können
%
\usepackage{rotating}




%
%1 Glossaries / Verzeichnisse
%
\usepackage[nonumberlist,toc,nopostdot]{glossaries}
% Entferne alphabetische Gruppierung des Glossars
\renewcommand{\glsgroupskip}{}
\makeglossaries


%
% Paket um LIstings sauber zu formatieren.
%
\usepackage[savemem]{listings}
\lstloadlanguages{TeX}



%
% Neue Umgebungen
%
\newenvironment{ListChanges}%
	{\begin{list}{$\diamondsuit$}{}}%
	{\end{list}}

%
% aller Bilder werden im Unterverzeichnis figures gesucht:
%
\graphicspath{{bilder/}}


%
% Anführungsstriche mithilfe von \textss{-anzufuehrendes-}
%
\newcommand{\textss}[1]{"`#1"'}

%
% Strukturiertiefe bis subsubsection{} möglich
%
\setcounter{secnumdepth}{4}

%
% Dargestellte Strukturiertiefe im Inhaltsverzeichnis
%
\setcounter{tocdepth}{4}

%
% Zeilenabstand wird um den Faktor 1.5 verändert
%
%\renewcommand{\baselinestretch}{1.5}
\usepackage{setspace} %Zeilenabstand einstellbar



\usepackage{caption}


\DeclareUnicodeCharacter{00A0}{ }



\usepackage{verbatim}




%\newcommand{\submissiondate}{27. August 2015}
%\newcommand{\submissionmonthyear}{August 2015}


% Tabellenprogramm
\usepackage{tabularx}
\usepackage{multirow}

% Plotten von Tabellen
\usepackage[svgnames]{xcolor}
\usepackage{pgfplots}
\pgfplotsset{every axis/.append style={
thick,
tick style={thick}},
%width=10cm,
compat=1.12,
cycle list={Bihler1\\Bihler2\\green\\orange\\},
}
\usepgfplotslibrary{external}
\usetikzlibrary{pgfplots.groupplots}



\usepackage{siunitx}



\addto\captionsngerman{
\renewcommand{\figurename}{Abb.}
\renewcommand{\tablename}{Tab.}
%\renewcommand{\refname}{Quellenverzeichnis}
% \renewcommand{\bibname}{Quellenverzeichnis}
}




% Bihlerfarben
\definecolor{Bihler1}{RGB}{21, 78, 113}
\definecolor{Bihler2}{RGB}{146, 46, 46}




\usepackage[section]{placeins}

\usepackage{pdfpages}

\usepackage{lscape}


\usepackage{colortbl}






\usepackage{microtype}
%\overfullrule=2cm



\newcommand*\cleartoleftpage{%
  \clearpage
  \ifodd\value{page}\hbox{}\newpage\fi
}





\begin{comment}
%
% Paket für Links innerhalb des PDF Dokuments
%
\definecolor{LinkColor}{rgb}{0,0,0.5}
\usepackage[%
	pdftitle={Titel},% Titel der Diplomarbeit
	pdfauthor={Jens Schmelkus},% Autor(en)
	pdfcreator={LaTeX, LaTeX with hyperref and KOMA-Script},% Genutzte Programme
	pdfsubject={Diplomarbeit}, % Betreff
	pdfkeywords={Keywords}]{hyperref} % Keywords halt :-)
\hypersetup{colorlinks=false,% Definition der Links im PDF File
	linkcolor=LinkColor,%
	citecolor=LinkColor,%
	filecolor=LinkColor,%
	menucolor=LinkColor,%
	pagecolor=LinkColor,%
	urlcolor=LinkColor}
	
	
\definecolor{Colour1}{HTML}{1B9E77}
\definecolor{Colour2}{HTML}{D95F02}
\definecolor{Colour3}{HTML}{7570B3}
\definecolor{Colour4}{HTML}{E7298A}
\definecolor{Colour5}{HTML}{66A61E}
\definecolor{Colour6}{HTML}{E6AB02}
\definecolor{Colour7}{HTML}{A6761D}
\end{comment}
