\chapter*{Zusammenfassung}\addcontentsline{toc}{chapter}{Zusammenfassung}

In dieser Arbeit wird der Betriebsbereich für den Tragflügel eines Unbemannten Kleinstflugzeuges Ausgelegt, sowie die Strukturellen Anforderungen und Mechanischen Grenzen Formuliert. Des weiteren werden die Kräfte und Momente für die Lastfälle Betrachtet welche kritisch für die Auslegung des Struktur sind.
Es erfolgt eine Beschreibung des bisherigen Baumethoden und ein Vergleich verschiedener im Modellbau üblichen Baumethoden.
Daraus werden die Verbesserungsmöglichkeiten und Problemstellungen Formuliert.
Es wurde ein Flügel mit 2,8 m Spannweite und 0,3 m Wurzeltiefe konzipiert und mit Traglinienmethoden in XFLR5 simuliert.
Die aus der Berechnung mit XFLR gewonnen Kräfte und Momente gehen in die Auslegung der Struktur ein.
Damit wurde ein Grobmodell der Halbspannweite des Tragflügels mit CAD Methoden erstellt.
