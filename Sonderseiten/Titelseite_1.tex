% Die Titelseite
% Im folgenden kommen ein paar Variablen, die auszufüllen sind
% Bisher steht dort nur Musterinhalt
% Außerdem müssen zei Dateien erstellt werden, Bild/Logo/Emblem des Fachgebietes
% sowie der Universität

\newcommand{\trtitle}{Entwickeln von Prüfkriterien und eines Testablaufes für die Abnahme von NC Aggregaten in der Serienfertigung}
\newcommand{\trtype}{Diplomarbeit}
\newcommand{\trauthor}{Jens Schmelkus}
\newcommand{\trstrasse}{Im Gässsle 1}
\newcommand{\trmatrikelnummer}{06097611}
\newcommand{\trort}{86869 Oberostendorf / Lengenfeld}
\newcommand{\trprof}{Prof. Dr.-Ing. Frank Palme}
\newcommand{\trbetreuer}{Thomas Grotz}
\newcommand{\trfachgebiet}{XXXX}
\newcommand{\trfakultaet}{Fakultät 03}
\newcommand{\truni}{Hochschule München}
\newcommand{\trdate}{\today}

\thispagestyle{empty}

% Kopfzeile mit Logos.
% Eventuell die \hspace{} je nach Logogröße anpassen
\begin{tabular}{lcr}
  \includegraphics[scale=0.08]{fk03} & % dein_unilogo.jpg/.eps im Verzeichnis "bilder" ablegen
  \hspace{1.5cm} \truni \hspace{1cm} &
  \trauthor
  \\
\end{tabular}

\rule{\textwidth}{0.8pt}

\vspace{1cm}
\begin{center}
  \textbf{\LARGE \trtitle}
\end{center}


\begin{center}
\includegraphics[scale=0.8]{Bihler_Logo} % dein_fglogo.jpg/.eps im Verzeichnis "bilder" ablegen, Fachgebietslogo
\\
\end{center}

\begin{center}
  \textbf{\trtype} \\
  %am Fachgebiet \trfachgebiet \\
  \trprof \\
  
  \trfakultaet \\
  \truni \\[0.5cm]
  vorgelegt von \\
  \textbf{\trauthor}
\end{center}

\vspace{1cm}



\begin{center}
\begin{tabular}{ll}
Betreuer: & \trbetreuer \\
\end{tabular}
\end{center}


\vfill

\begin{tabular}{l}
\trauthor \\
Matrikelnummer:  \trmatrikelnummer \\
\trstrasse \\
\trort
\end{tabular}

\rule{\textwidth}{0.4pt}