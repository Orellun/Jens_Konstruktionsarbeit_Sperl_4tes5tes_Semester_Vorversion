\chapter*{Nomenklatur}
%\markboth{Nomenklatur}{Nomenklatur}
\addcontentsline{toc}{chapter}{Nomenklatur}
%
% In diesem Abschnitt werden alle in diesem Dokument verwendeten Bezeichner beschrieben.

% Falls zur jeweiligen Größe kein Koordinatensystem gegeben ist, wird dieses als Index spezifiziert oder ist nicht zutreffend.

%\section*{Kleine lateinische Bezeichner}
%\vspace{0.2cm}\noindent
%\begin{tabularx}{\textwidth}{lX}
 %$a$ & Schallgeschwindigkeit\\
%\end{tabularx}
%\vspace{0.5cm}\noindent


\section*{Formelzeichen}

\begin{tabularx}{\textwidth}{lll}
\textbf{Symbol} & \textbf{Einheit} & \textbf{Beschreibung}\\
& & \\
$A$ & \si{\meter\squared} & Querschnittsfläche \\
$D$ & \si{\milli\meter} & Durchmesser \\
$f$ & \si{\hertz} & Frequenz \\
$f(z)$ & 1 & normiertes Bewegungsgesetz \\
$J$ & \si{\kilogram\meter\squared} & Massenträgheitsmoment \\
$k_\mathrm{v}$ & \si{\meter\squared} & Durchflussbeiwert  \\
$K_v$ & \si{\cubic\meter\per\hour} & Durchflussfaktor \\
$M$ & \si{\newton\meter} & Moment \\
$n$ & \si{1\per\minute} & Drehzahl \\
$p$ & \si{\milli\meter} & Spindelsteigung \\
$p$ & \si{\bar} & Druck \\
$Q$ & \si{\cubic\meter\per\hour} & Dichte \\
$s(\varphi)$ & \si{\milli\meter\per\radian} & Weg des Abtriebsgliedes (Übertragungsfunktion 0. Ordnung) \\
$S$ & \si{\milli\meter} & Gesamtweg des gerade geführten Abtriebsgliedes \\
$t$ & \si{\second} & Zeit \\
$z$ & 1 & normierter Drehwinkel eines Bewegungsabschnittes \\
$Z$ & 1 & Anzahl \\
$\zeta$ & 1 & Druckverlustbeiwert \\
$\rho$ & \si{\kilogram\per\liter} & Dichte \\
$\varphi$ & \si{\radian} & Drehwinkel des Kurvenkörpers \\
$\Phi_{ik}$ & \si{\radian} & Gesamtdrehwinkel des Kurvenkörpers im Abschnitt $ik$  \\
\end{tabularx}

\clearpage

\section*{Indizes}

\begin{tabularx}{\textwidth}{ll}
\textbf{Symbol} & \textbf{Beschreibung}\\
& \\
$A$ & Außenring \\
$G$ & Gewinderollentrieb \\
$I$ & Innenring \\
$ik$ & Nummerierung der Bewegungsabschnitte \\
$max$ & Maximal \\
$n$ & Drehzahl \\
$NCA$ & Auf das Numerical Controlled Aggregat bezogen \\
$Ring$ & Ringkontakt am Wälzkörper \\
$T$ & Teilkreis \\
$ue$ & Überrollfrequenz \\
$W$ & Wälzkörper \\
$WA$ & Wälzkörper \\
$zus$ & zusätzlich \\
\end{tabularx}



\begin{comment}

\vspace{0.2cm}\noindent
\begin{tabularx}{\textwidth}{lll}
Symbol & Einheit & Beschreibung\\
$t$ & \si{\second} & Zeit \\
$p$ & \si{\milli\meter} & Spindelsteigung \\
$f(z)$ &  & Normierte Übertragungsfunktion \\
$n$ & \si{1\per\minute} & Drehzahl des Hauptantriebs \\
$a$ & \si{1\per\second\squared} & Winkelbeschleunigung \\
$u_{max}$ & \si{1\per\minute} & Maximale Drehzahl des NCA's \\
$S_{ik}$ & \si{\milli\meter} & Gesamtweg des gerade geführten Abtriebsgliedes im Abschnitt $ik$ \\
$\omega_a = \dot{\phi}$ & \si{\radian\per\second} & Winkelgeschwindigkeit des Kurvenkörpers \\
$\Phi_{ik}$ &   & Gesamtdrehwinkel des Kurvenkörpers im Abschnitt $ik$  \\
$ik$ & \si{\radian} bzw. \si{\degree} & Nummerierung der Bewegungsabschnitte \\
$z$ & 1 & normierter Drehwinkel eines Bewegungsabschnittes \\
$\varphi$ & \si{\radian} bzw. \si{\degree} & Drehwinkel des Kurvenkörpers \\
$s(\varphi)$ & \si{\milli\meter\per\radian} & Weg des Abtriebsgliedes (Übertragungsfunktion 0. Ordnung) \\
$n_{max}$ & \si{1\per\minute} & Maximale Drehzahl der Maschine \\
\end{tabularx}



 

\colorbox{orange}{nicht Aktuell muss noch überarbeitet werden}


\end{comment}




%  $u_{max}$ in $\frac{\text{U}}{min}$
 
 
 % TODO: Indizes