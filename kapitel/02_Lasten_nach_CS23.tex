\chapter{Bestimmung der Lasten nach CS 23}\label{cha:Bestimmung der Lasten nach CS 23}

\section{Ausgangsannahmen für das Fluggerät}

\subsection{Gewichte und Dimensionen}

Für die Berechnung der Lasten und Sicherheiten wird von Folgenden Grundannahmen für das FLugzeug ausgegangen:

Tabelle

\subsection{Der Tragflügel}

Der Tragflügel wird für eine Flugaufgabe mit folgenden Paramteren ausgelegt:

Tabelle:

Fluggeschwindikeit 15 m/s

Spannweite in Teilung von 700 mm aus transportgründen

Verwendung des bewährten Profils Clark Y

Rechteckfläche mit angeschlossenem Trapez

Hierbei werden Eigenschaften von bisher Eingesetzten Flügel übernommen.
Eine Disskussion über die Wahl dieser Aerodynamischen Parameter findet in dieser Arbeit nicht statt. 
Es wird von der Strukturoptimierung ausgegangen.
Die Wahl der geradlinig aufgebauten Tragflächensegmente ermöglicht hier Vorteile bei der Fertigung in beiden Bauweisen.


Daraus resultiert ein Tragwerk mit einer Gesamtflügelfläche von blaa bei einer Spannweite von blaa

Im Bild blaa ist ein Aufriss mit den Grundabmessungen dargestellt.

\section{Bestimmung der Lasten}

Mit seinem Gewicht von 4,5 Kg fällt der Flieger nach CS 23 in die Kategorie blub
Alle Auslegungsrechnungen basieren auf dieser Einordnung

\subsection{Lasten im statischen Flug}

\subsection{Lasten im Maximalfall}

Abfangen

Nach CS 23.15 liegt die Maximale last für den Abfangfall bei 1.43 des Maximalen Abfluggewichts

Hochstart

Hier kann die Konservativere Auslegung der Lasten gegebenfalls für den belastendensten Fall beim Hochstart mittels Gummiseil abgeschärtzt werden. Hier übersteigt die Auftriebsbelastung für den Hochanstellwinkelbereich kurzeigtig den Maximalauftrieb des Probfils durch den Betrieb im Instationären bereich.