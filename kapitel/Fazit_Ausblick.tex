\chapter{Fazit und Ausblick}\label{cha:Fazit und Ausblick}

In den letzten Jahren hat sich die Elektronik des Automatischen Flugsystems des Labor für Systemtechnik sich seinen vielfältigen Aufgaben angepasst.
Dabei hat sich sowohl die Augabenformulierung und das Konzept von ersten Versuchen nahe einem klassischen Modellflugzeug zu einem vollständigen System mit Platinenaufbau und zahlreichen Unteraufgaben gewandelt. Es ist ein vielfältiges Repertoire an erprobten und gereiften Elektronikbaugruppen entstanden welche sich in stetiger Optimierung befinden. Damit ist sowohl eine schnelle Anpassung der Gesamten Elektronik an neue Aufgaben und Dimensionierungen möglich.
Damit hat die Elektronik die vorausgegangene Entwicklung des mechanischen Aufbaus nachvollzogen, welche auch eine gezielte Verbesserung einzelner Funktionen bei begrenztem Arbeitsaufwand ermöglicht.
Ebenfalls ist mit den gewachsenen Erfahrungen mit dem Einsatz des Flugsystems im Zusammenspiel mit den Elektronischen Schutzmaßnahmen ein hohes Sicherheitsniveau für Bedienmannschaft und Flugzeug gewährleistet. Dies hat sich besonders im oft hektischen Wettbewerbseinsatz als wertvoll erwiesen.

Wie das Kapitel der weiteren Entwicklungen bereits aufzeigt entstehen auch Abseits der eigentlichen Schaltungsentwicklung neue Ansätze und Aspekte um Handhabung und Leistungsfähigkeit des Gesamtsystems zu steigern.

Somit wird das Elektronische Gesamtsystems des Automatischen Flugsystems des Labor für Systemtechnik auch in den kommenden Jahren konkurrenzfähig auf dem AUVSI Wettbewerb antreten können, als sich auch kurzfristig in Forschungseinsätzen bewähren.

